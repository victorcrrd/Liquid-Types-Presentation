\section{Case Study: OCaml and DSOLVE}

% DSOLVE diagram
%\begin{frame}{Case Study: Ocaml and DSOLVE}
%\begin{figure}[h]
%\includegraphics[width=8cm]{figures/dsolve_diagram.png}
%\end{figure}
%\end{frame}

% Examples
% slide 1
\begin{frame}[containsverbatim]{Examples .ml}

\begin{lstlisting}[language=Caml]
let max x y =
  if x > y then x else y
\end{lstlisting}

\begin{lstlisting}[language=Caml]
let rec sum k =
  if k < 0 then 0 else
     let s = sum (k-1) in
     s + k
\end{lstlisting}

\end{frame}

% slide 2
\begin{frame}[containsverbatim]{Examples .ml}

\begin{lstlisting}[language=Caml]
let foldn n b f = 
  let rec loop i c =
    if i < n then loop (i+1) (f i c) else c int
  loop 0 b
\end{lstlisting}

\begin{lstlisting}[language=Caml]
let arraymax a = 
  let am l m = max (sub a l) m in 
  foldn (len a) 0 am

\end{lstlisting}
\end{frame}

% slide 3
\begin{frame}[containsverbatim]{Examples .hquals}
\begin{lstlisting}[language=Caml]
qualif POS(v): 0 <= v 
qualif LT(v): ~A <= v
qualif GT(v):  v < ~A
qualif BND(v): v < Array.length ~A 
\end{lstlisting}
\end{frame}


% slide 3
\begin{frame}[containsverbatim]{Examples liquid types}
\begin{verbatim}
max:: x:int -> y:int -> {v:int | (x <= v) && (y <= v)}

sum:: k:int ->{v:int | 0 <= v && k <= v}

foldn :: forall a.
    n:int -> 
    b:a -> 
    f:({0 <= v || v < n} -> a -> a) -> a

arraymax :: intarray -> {v:int | 0 <= v}
\end{verbatim}
\end{frame}